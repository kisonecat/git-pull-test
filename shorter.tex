\documentclass{ximera}

\newcommand{\RR}{\mathbb R}
\renewcommand{\d}{\,d}
\newcommand{\dd}[2][]{\frac{d #1}{d #2}}
\renewcommand{\l}{\ell}
\newcommand{\ddx}{\frac{d}{dx}}
\newcommand{\dfn}{\textbf}
\newcommand{\eval}[1]{\bigg[ #1 \bigg]}

%%%%%%%%%%%%%%%%%%%%%%%%%%%%%%%%%%%%%%%%%%%%%%%%%%%%%%%%%%%%%%%%
% Here is an example of building a new HTML div
\newenvironment{blinkblock}{}{}
\newcommand{\blink}[1]{#1}

\outcome{}

\newcommand{\RS}{r^s}

\title[Break-Ground:]{Short thing}

\begin{document}

\ifdefined\HCode
\ConfigureEnv{blinkblock}{\ifvmode \IgnorePar\fi \EndP\HCode{<div class="blink">}}{\HCode{</div>}\IgnoreIndent}{}{}
\renewcommand{\blink}[1]{\HCode{<span class="blink">}#1\HCode{</span>}}
\fi

\begin{abstract}
Here we see a dialogue where two young mathematicians discuss limits
and instantaneous velocity.
\end{abstract}

\maketitle

So I wanted to demonstrate building a new div from within TeX.

Here's some \blink{blinking text}.  I miss the \blink{blink tags}.

\begin{problem}

Let's try something easy, like $1 = \answer{1}$.

\begin{problem}
If $A = 1 + 1$, then $A = \answer{2}$.

\begin{hint}
Think about two objects.
\end{hint}

\begin{hint}
Seriously, think about two objects.
\end{hint}

\begin{hint}
Think more.
\end{hint}
\end{problem}

\begin{problem}
If $B = 5 + 3$, then $B = \answer{8}$.

\begin{hint}
Think about eight objects.
\end{hint}

\begin{hint}
Seriously, think $4 + 4 = \answer{8}$.
\end{hint}

\begin{hint}
Think ATE.
\end{hint}
\end{problem}

\begin{problem}
If $C = 3 + 3$, then $C = \answer{8}$.
\end{problem}

\end{problem}

\hrule

\begin{problem}
  Let's try multiplying.  Recall $3 \times 2 = \answer{6}$ and $3 \times 4 = \answer{12}$.
  
  \begin{problem}
    If $A = 1 \cdot 1$, then $A = \answer{1}$ and $A - 1 = \answer{0}$ and $A^2 + 2= \answer{2}$.
    
    \begin{problem}
      And then $A/2 = \answer{1/2}$.
      
      \begin{problem}
        And then $A/4 = \answer{1/4}$.
      \end{problem}
    \end{problem}
  \end{problem}
  
  \begin{problem}
    If $B = 5 \cdot 3$, then $B = \answer{15}$.
    
    \begin{problem}
      So $B + 1 = \answer{16}$.
    \end{problem}
  \end{problem}
  
  \begin{problem}
    If $C = 3 \cdot 3$, then $C = \answer{9}$.
    
    \begin{problem}
      So $C + 1 = \answer{10}$.
    \end{problem}
  \end{problem}
  
\end{problem}



\end{document}
